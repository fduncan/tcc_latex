\chapter{INTRODUÇÃO}

A atual conjuntura econômica mundial configura um ambiente de extrema competição, onde os mecanismos de diferenciação se interagem, condicionando o desempenho das empresas em seus respectivos mercados. Nesse contexto, a busca da eficiência e eficácia é uma exigência que se impõe a todos os processos e níveis organizacionais presentes nas atividades produtivas. Para não perderem suas fatias de mercado, as empresas tendem a se posicionar mais agressivamente nessa disputa. Nesse caso, a grande organização muitas vezes leva vantagem por estar mais bem estruturada.

À medida em que a competição se torna mais acirrada, maior é a importância dos ganhos de produtividade trazidos pela tecnologia da informação. Essa tendência influencia toda a cadeia produtiva, o que faz prever que qualquer empresa, independente de seu porte ou tipo de atividade, terá que considerar os impactos que seus softwares trarão para seus negócios, seu mercado e sua concorrência. As empresas estão utilizando cada vez mais os softwares como ferramenta de competitividade, com impactos importantes e positivos nos seus negócios, nos mais variados ramos de atividade.

Percebeu-se que com a mudança constante do mercado ou até mesmo nas próprias empresas, os softwares precisam acompanhar estas mudanças, com a maior rapidez possível, garantindo assim qualidade, mantendo confiabilidade, eficiência e escala de produção. Controlar a qualidade de sistemas de software é um grande desafio devido à alta complexidade dos produtos e às inúmeras dificuldades relacionadas ao processo de desenvolvimento, que envolve questões humanas, técnicas, burocráticas, de negócio e políticas. Idealmente, os sistemas de software devem, não só fazer corretamente o que o cliente precisa, mas também fazê-lo de forma segura, eficiente, escalável, flexível e de fácil manutenção e evolução \cite{OLIVEIRA}. Com isso, surge a idéia dos testes automatizados, que fazem com que o custo e o risco do software caiam consideravelmente, aumentando a confiabilidade, pois com estes o desenvolvedor tem certeza que o software faz exatamente o que deveria fazer \cite{MYERS}. A grande vantagem desta abordagem, é que todos os casos de teste podem ser facilmente e rapidamente repetidos a qualquer momento e com pouco esforço.

A reprodutibilidade dos testes permite simular identicamente, e quantas vezes for desejada, situações específicas, garantindo que passos importantes não sejam ignorados por falha humana, facilitando a identificação de um possível comportamento não desejado \cite{DELAMARO}.

Além disso, como os casos para verificação são descritos através de um código interpretado por um computador, é possível criar situações de testes bem mais elaboradas e complexas do que as realizadas manualmente, possibilitando qualquer combinação de comandos e operações \cite{DELAMARO}. Utilizando testes automatizados, é possível simular centenas de usuários acessando um sistema ou inserir milhares de registros em uma base de dados, o que não é factível com testes manuais.

Todas estas características ajudam a solucionar os problemas encontrados nos testes manuais, diminuindo a quantidade de erros e aumentando a qualidade do software. Como é mais rápido e fácil executar todos os testes a qualquer momento, mudanças no sistema podem ser feitas com segurança, o que aumenta a vida útil e qualidade do produto. Com isso surgiu a idéia da Integração Contínua, que consiste em uma prática em que os desenvolvedores de uma equipe integram seu trabalho pelo menos uma vez por dia, o que consequentemente levará a múltiplas integrações do sistema \cite{FOWLER}, facilitando assim a detecção de erros inesperados \cite{DUVALL}.

\section{Justificativa do trabalho}

A Integração Contínua foi escolhida como tema deste trabalho por se tratar de uma técnica recente, com poucas referências teóricas que abordam especificamente o assunto. Além disso, o desejo de criar um material que se torne uma referência, em português, para que outras pessoas possam estudar sobre o assunto.

\section{Objetivo}

O principal objetivo deste trabalho é, através de um estudo de caso, documentar e implantar a Integração Contínua em um projeto real. Almeja-se, também, fornecer ao leitor deste trabalho, conhecimento técnico sobre a aplicação dos conceitos da Integração Contínua. Desta forma, deseja-se descrever estes conceitos para equipes de desenvolvimento de software que vislumbram os benefícios desta técnica. Ademais, aumentar a confiabilidade do projeto da Biblioteca Digital da RENAPI, visando um projeto mais robusto e menos suscetível a erros.

\section{Estrutura do trabalho}

Este trabalho se divide em 5 capítulos e está estruturado da seguinte maneira:

No capítulo 2 é feita uma fundamentação de conceitos e técnicas necessárias para o entendimento deste trabalho.

No capítulo 3 é feita uma descrição da Integração Contínua, seus componentes, suas práticas e suas vantagens.

Já no capítulo 4, é apresentado um estudo de caso, em que os autores deste trabalho são responsáveis por implantar um ambiente de Integração Contínua, em um projeto real de desenvolvimento de software.

E por fim, o capítulo 5 apresenta os resultados obtidos com o estudo de caso bem como os trabalhos futuros a serem desenvolvidos no processo da Integração Contínua implantado no projeto.

