\chapter{CONCLUSÕES}

\section{Objetivos alcançados}

Com o trabalho realizado foi possível documentar a implantação da Integração Contínua no projeto da Biblioteca Digital da RENAPI.

Com o ambiente de Integração Contínua implantado, mesmo que incompleto, percebeu-se que os desenvolvedores ficam mais confiantes no projeto, visto que a base de código não fica inconsistente por muito tempo. Com toda suíte de testes sendo executada, a confiabilidade do projeto aumenta, bem como a qualidade do código desenvolvido.

Além disso, a documentação criada acerca da teoria da Integração Contínua e da descrição do estudo de caso, criam um bom material para conhecimento da prática e análise de um ambiente de integração em um projeto real.

\section{Trabalhos futuros}

Como o modelo Síncrono de integração é mais confiável e apresenta menos desvantagens do que o modelo Assíncrono, pretende-se utilizar o modelo de Integração Contínua Síncrono ao invés do modelo de Integração Contínua Assíncrono, bastando para isso, disciplinar a equipe para nunca deixar a base de código instável e obter um símbolo de integração para mostrar a equipe quem está integrando naquele exato momento.

Ademais, de acordo com o que foi mencionado na sessaõ 4.8 - Problemas remanescentes, almeja-se separar os testes unitários dos testes de aceitação para agilizar o processo de \textit{build}. Para dinamizar a integração, os testes de aceitação podem ser executados somente uma vez no dia, enquanto os testes unitários devem ser executados a cada \textit{commmit}. Com a redução do tempo do \textit{build}, almeja-se também, fazer a integração de todos os módulos do sistema.
